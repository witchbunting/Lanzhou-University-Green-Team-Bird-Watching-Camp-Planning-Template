
\documentclass[UTF8,a4paper,12pt]{ctexart}
\usepackage{zhnumber} 
\usepackage{color}
\usepackage[left=2.50cm, right=2.50cm, top=2.50cm, bottom=2.50cm]{geometry} %页边距
\usepackage{ctex}
\usepackage{helvet}
\usepackage{amsmath, amsfonts, amssymb} % 数学公式、符号
\usepackage[english]{babel}
\usepackage{enumitem}
\usepackage{graphicx}   % 图片
\usepackage{url}        % 超链接
\usepackage{bm}         % 加粗方程字体
\usepackage{multirow}
\usepackage{booktabs}
\usepackage{indentfirst}\setlength{\parindent}{2em}
\usepackage{float}
\usepackage{listings} 
\usepackage{algorithm}
\usepackage{algorithmic}
\renewcommand{\algorithmicrequire}{ \textbf{Input:}}       
\renewcommand{\algorithmicensure}{ \textbf{Initialize:}} 
\renewcommand{\algorithmicreturn}{ \textbf{Output:}} 
\renewcommand\thesection{\zhnum{section}}
\renewcommand \thesubsection {\arabic{section}}
\usepackage{hyperref}
\hypersetup{hidelinks}
\ctexset{
	subsection={
	    number = \arabic{subsection},
		aftername = \hspace{1em},
		%设置缩进无用
		indent = {1.5em}
	}
	}

\linespread{1.3}
\usepackage{fancyhdr} %设置页眉、页脚
\pagestyle{fancy}
\CTEXsetup[format={\Large\bfseries}]{section}
\setlength{\arraycolsep}{1pt}
\fancyhead[L]{
	\begin{minipage}[c]{0.06\textwidth}
		\includegraphics[height=12mm]{image/logo.png}
	\end{minipage}}
	\chead{兰大绿队,我们的家\qquad 倡导绿色生活,让改变发生}
\lfoot{地址:兰州大学榆中校区院办 25 号楼 418 室\\网址:\textcolor{blue}{\url{www.greenteam.org.cn}}\\邮政编码: 730107}
\rfoot{E-mail: ldgreenteam@163.com}
\begin{document}\large
\begin{titlepage}
    \begin{figure}
        \centering
        \includegraphics[width=0.6\linewidth]{image/title.png}
    \end{figure}
\begin{center}
    \fontsize{25pt}{0}\bf{兰州大学绿队第十五届观鸟营}\\
    \vspace{0.8em}
    \fontsize{15pt}{0}\bf{这里写调研名称}\\
    \vspace{1.5em}
    \fontsize{52pt}{0}\bf{策}\\
    \vspace{1.5em}
    \fontsize{52pt}{0}\bf{划}\\
    \vspace{1.5em}
    \fontsize{60pt}{0}\bf{书}\\
    \vspace{2em}
    \fontsize{15pt}{0}\bf{主办方:兰州大学绿队}
\end{center}
\end{titlepage}
\clearpage
\pagestyle{fancy}
\tableofcontents 
\newpage
\setcounter{page}{1}
\vspace{1.5em}
\rhead{Page \thepage}
\newpage
\section{项目简介}
\subsection{项目名称}
\hspace{3pt} 兰州大学第十五届观鸟营
\subsection{项目宗旨}
\hspace{3pt} 
\subsection{项目地点}
\hspace{3pt} 这里填写项目地点
\subsection{时间}
\hspace{3pt} time
\subsection{负责人}
\hspace{3pt} 尹张驰(女巫)......
\section{项目背景}
\subsection{观鸟活动及意义}
观鸟指人们利用望远镜等光学设备, 在尽可能避免惊扰鸟类的情况对自然状态下的野
生鸟类及其生存环境进行观察和记录。 观鸟活动的内容包括观察鸟类的形态、 鸣叫等特征
并据此辨别鸟的种类,观察鸟类的取食、 栖息、 繁殖、 迁徙等行为,了解鸟类与其生存环境
的关系、 在一定区域内鸟类种群群落的动态变化等等。 观鸟最大的意义是在观鸟过程中完
成的观鸟记录, 能够为鸟类学基础研究搜集必要数据, 为野生鸟类学研究提供一个巨大的
数据库, 起到科学家眼线的作用。 鸟类的形态丰富多彩, 活泼好动,我们通过参与观鸟活动
还可以进一步亲近自然, 放松身心。 从观鸟开始, 培养自然环保的生态理念。 此外, 观鸟
活动可以在不影响自然环境的前提下为经济欠发达的地区提供一定的旅游收入。 观鸟旅游
是生态旅游重要的表现形式之一, 是绿色无污染的产业, 对环境影响的负作用最小。 观鸟
活动作为“一张亲近自然的免费门票”, 使人们认识自然、 保护自然, 能够促进人与自然和
谐发展, 有效加强参与者保护自然意识。
\subsection{鸟类栖息地及生存状况}
野生动物栖息地选择的研究是动物学研究的一个基本而又重要的领域, 鸟类对栖息地
的选择以植被类型为基础。 在不同空间尺度上, 影响鸟类栖息地选择的环境因素亦不相同:
在不同的季节和生活史阶段, 鸟类对栖息地类型的选择取决季节性变化的植被因素和鸟类
在不同生活史阶段对栖息地的不同需求; 而鸟类的繁殖栖息地选择取决于小尺度上的植被结构。 根据栖息地的不同, 鸟类可被分为林灌鸟类、 开阔区鸟类、 水域鸟类和沼泽鸟类。
当前, 随着经济的快速发展, 人类活动己经严重破坏鸟类赖以生存的栖息地, 全球鸟类存
状况堪忧。 国际鸟类联盟最新一份研究报告指出, 全球八分之一的鸟类——1300 多种鸟,
面临着灭绝危机, 它们的生存状况日益恶化, 从热带地区至两极地区, 一些鸟类物种的数
量呈现可怕的下降趋势。
\subsection{目的地介绍}
[0.5-1]
\section{项目目标}
\begin{enumerate}[itemindent=1em]
    \item 通过前期资料收集查询、 中期实地采访调研、 后期总结思考并形成完整相关报告的方式, 考察当地生态文明建设分析报告书。
    \item 争取与自然保护区合作, 调查自然保护区的鸟种, 完成鸟种统计并上传观鸟记录,
并对比当地文献资料, 对鸟种情况进行统计分析。
    \item 通过观鸟、 植物认知、 观鸟写生等自然体验活动, 提高营员的自然体验感, 加深
对自然的认识, 并培养出观鸟水平较高的小孩。
    \item 组织营员进行各自社团活动介绍, 促进各环保社团建设经验交流与分享, 借鉴其
他社团的长处以促进绿队的发展。
    \item 借助兰州大学绿队微信公众号等新媒体平台宣传活动及相关成果, 扩大活动的影
响范围, 让更多的人参与观鸟, 了解鸟类, 进而保护鸟类、 保护野生动物。
\end{enumerate}
\section{项目内容和实践方式}
\subsection{前期准备}
\begin{enumerate}[itemindent=1em]
    \item  资料查询与准备\\踩点......
    \item  出营前培训\\
        \begin{enumerate}[itemindent=1em]
            \item 营期常见鸟类基础培训:\\观鸟入门培训、 鸟类身体部位及六大生态类群、 营期常见鸟类分科简单培训。
            \item 每日鸟类打卡:\\每天带队会在群内发一张鸟图或出其他题目, 营员需要在当日 00: 00 前进行辨认并将辨认结果以“鸟种名+辨认理由” 发送给当日发布认鸟公告的带跟队。\\同时在次日 12:00 前在群内进行接龙打卡。
            \item 本地营员野外观鸟培训:\\在选拔出本地营员后,会每周带领本地营员进行校园观鸟或者萃英山观鸟等活动, 旨在培养观鸟兴趣、 提升观鸟能力, 更好的体验自然、 融入自然。
            \item 户外急救知识培训:\\由于营期的活动主要集中在户外,而且地区海拔较高,在开营前需要通过文字、 视频等方式向成员科普急救知识和药品用法, 以备不时之需。
        \end{enumerate}
    \item 宣传与招募\\
        对本次营期通过线上和线下两个渠道进行宣传。 线下宣传即在视野广场张贴海报。 线
上宣传主要包括绿队官方微信等网络平台对营期进行大力宣传。
    \item 线路与住宿规划\\
        内容......
\end{enumerate}
\subsection{中期活动}
\begin{enumerate}[itemindent=1em]
    \item 自然体验——观鸟:\\在观鸟是营期的主要活动内容, 通过带领成员不同地区不同路线的野外观鸟和分组比
赛等形式, 提升成员对鸟类的认识、 增强保护鸟类的意识。
\item 调研访谈:\\主要包括对当地旅客、 当地居民以及当地政府的访问与调研, 借此了解当地对爱护鸟
类的态度与行动, 以及当地生态文明建设的措施与成果。
\item 公共宣讲:\\自新冠肺炎以后, 我国人民对野生动物的保护意识有了较为明显的提高, 通过宣传鸟
类保护知识与急救措施, 将爱鸟护鸟意识深入人心, 响应保护野生动物的号召。 并借此机
会宣传兰大鸟类文创产品。
\item 绿地图:\\观鸟绿地图的制作能增强营员对观鸟路线的熟悉以及培养营员逐渐了解大自然的能
力。 充满童趣的手绘, 色彩斑斓的拼接, 各种物种的分布以及地区的介绍, 使得绿地图作
为营期宣传作品为大家熟知。
\item 观鸟比赛:\\通过分组观鸟比赛的形式, 增强竞争性以及趣味性, 旨在提升营员的观鸟能力与团队
合作意识。
\item 社团交流:\\通过聆听各个社团之间的故事与体验, 增进营员彼此之间的了解, 学习其他社团优秀
的活动与准则, 完善彼此、 促进发展。
\item 国王与天使:\\国王与天使这项游戏通过天使守护国王的基本规则加强了营员之间的交流, 促进团队
成员更快熟络起来, 有益于培养感情。
\item 每日双星与总结:\\每日总结是营期活动中必不可少的环节, 在结束一天的行程后, 营员们聚集在一起总
结下今天一天的感受并提出反馈, 可以及时规避或者改正错误; 每日双星通过每日抽取两
位营员讲故事或者人生经历等, 加快营员的亲近感与熟悉度。
\end{enumerate}
\subsection{后期整理}
\begin{enumerate}[itemindent=1em]
\item 文集制作:\\ 回到兰州, 营员进行文集制作, 保留珍贵的营期回忆。
\item 宣传视频制作:\\ 利用营期中拍摄的视频, 通过剪辑、 合成等制作成一份宣传视频供
下一届营期宣传。
\item 成果整理:\\ 撰写关于连城生态文明建设的分析报告, 将结果报告、 鸟种调研记录、
绿地图等反馈给当地保护区, 促进交流发展。 同时将绿地图、 调查记录等营期成果进行分
类整理。 将绿地图收集好可用于宣传, 调查结果整理成电子版上传到相关网站。
\end{enumerate}
\section{项目流程控制}
\subsection{准备阶段}
\begin{table}[H]
\resizebox{\textwidth}{!}{%
\begin{tabular}{|l|l|l|l|}
\hline
时间 & 地点 & 内容 & 备注 \\ \hline
2 月下旬 & 线上 & 观鸟营筹委招募 &  \\ \hline
3 月 & 线上 & 撰写并修改策划 &  \\ \hline
4.1-4.30 & 榆中校区 & \begin{tabular}[c]{@{}l@{}}策划修改与审批,资金申\\ 请\end{tabular} &  \\ \hline
4 月下旬 & \begin{tabular}[c]{@{}l@{}}榆中校区、及周边\\ 地区\end{tabular} & 第一次踩点 & \begin{tabular}[c]{@{}l@{}}本次踩点主要目的是为了确立居住地、统计住宿\\ 费用和车程费用。\end{tabular} \\ \hline
5.1-5.6 & 青海 & 第一次踩点 &  \\ \hline
5.6-5.9 & 线上 & 本地营员招募 &  \\ \hline
6.9 & 线上 & 发布外地营员招募信息 &  \\ \hline
6.9-6.24 & 线上 & 外地营员报名 &  \\ \hline
7.5-7.7 & 连城自然保护区 & 第二次踩点 & 记录鸟种状况 \\ \hline
\end{tabular}%
}
\end{table}
\subsection{实施阶段}
table......
\section{预期结果}
\begin{enumerate}[itemindent=1em]
\item 提高营员的观鸟水平和环保意识,增进营员之间的交流,培养营员团结合作精神及合作意识。
\item 以野生动植物保护观念为切入点,引发人们对生态保护和美丽中国的思考,同时引起社会相关人士对这项活动的注意及支持,有望为以后的活动举办提供帮助,提升社会影响力。
\item 通过社团交流分享,了解其他高校社团的构架以及日常事务管理,为本社团的发展提供切实可行的建议。
\item 内容
\item 成功培养出观鸟水平较高的小孩。
\item 拍摄图片、制作视频以及公众号推送,完成营期文集、文创设计。
\item 推进高校学生的观鸟活动,推动观鸟活动在西北地区的普及,传播爱鸟护鸟理念。
\item 完成一篇关于当地生态文明建设的论文,并根据营员自身体验以及当地采访结果提出可行性建议。
\end{enumerate}
\section{项目风险预测及其预防成果}
table......
\section{项目监测与评估}
\subsection{项目监测}
每天以例会形式监督项目各项内容的进展, 分享心得体会, 并对第二天的活动做好安
排, 若营期有重大失误, 当天开集体反思会以纠正项目进程。 并向指导老师汇报项目的进
展状况, 请老师给出指导性建议。

\subsection{项目评估}
\begin{enumerate}[itemindent=1em]
\item 撰写项目相关论文和调查报告并提交。
\item 在保护区进行宣传时进行人数估计, 确定影响范围。
\item 对观鸟资料(包括观鸟鸟种记录和所拍影像资料等) 每天进行统计。
\item 对当地采访进行记录并汇总, 并查阅保护区和林业局相关文献资料, 及时进行记录和整理。
\item  联系相关媒体对于我们的宣传和调查进行相关报道。
\item  每日在 QQ、 微信等平台上发送项目进程。
\end{enumerate}
\section{项目可行性分析}
\begin{enumerate}[itemindent=1em]
\item 从 2007 年开始绿队队员分别赴全国大学生绿色营及各省绿色营、 各地观鸟营、 自
然体验营进行学习交流, 与各高校观鸟组织联系密切。
\item 观鸟组以榆中校区为基础开展了一系列观鸟的活动, 已经培养出掌握了一定观鸟
知识的成员。 观鸟营带跟队大部分具有观鸟营经验, 经验丰富。
\item 绿队从 2008 年到 2021 年顺利举办了 14 届观鸟营, 每一期结束都进行了总结,
并取得了一定的成果, 积累了大量的经验, 为本期项目的开展提供了宝贵的资料和坚实的
基础。
\item 兰大绿队观鸟营作为与厦大观鸟营、 浙大观鸟营、 华中观鸟营齐名的全国四大高
校观鸟营之一, 具有一定的知名度和影响力。
\item 兰州大学作为甘肃省唯一的 985 重点大学, 历史悠久, 声名远扬。 进行的暑期实
践, 可以更好的得到政府的支持, 从而更有利于观鸟营在当地的开展。
\item 观鸟营成员来自不同学院、 不同年级, 具有不同技能。
\item 其他......


\end{enumerate}
\section{项目资金与预算}
table......
\newpage
\noindent\textbf{\Large{附录}}
\appendix
\addcontentsline{toc}{section}{Appendices}
\section*{项目团队组织方案}
本届观鸟营总队员初定为 18人,项目筹备人员4人,带队队长参与策划的撰写、前期准备工作、后期总结。项目中期带队队长负责项目的具体实施, 其余跟队负责联络与内容整理。本地人员 10 人,外地交流人员 4 人。
在已有队员中通过选举的方式产生安全委员、伙食委员、财务委员等干事,并落实其职责,确保项目能够在有纪律和制度保障的前提下进行。
\begin{enumerate}[itemindent=1em]
\item 纪律委员(1 个):负责管理、维持营期的纪律,记录每个人的惩罚,及时监督营员做惩罚。并且负责团队所有队员的安全,需要做到时刻关注队友,关注团队。
\item 伙食委员(2 个):负责团队伙食的安排和饮食营养的搭配。需要有一定的做饭基础,伙食委员之间要相互协调。
\item 文娱委员(1 个):调动团队的气氛,使团队能够在紧张的行程中能够有机的放松,从而获得更强的“战斗力”,需要具有幽默的言语表达能力和充分的准备;需要负责营歌与营舞的教授。
\item 学习委员(2 个):负责记录每日团队的行程和时间、鸟种记录, 需要细心且认真;负责处理团队各种文献资料的整理等工作,需要有默默无闻的奉献精神。
\item 医疗委员(1 个):负责项目中期队员出现疾病的预防及简单治疗处理,能够保证队员身体正常并能够做在出现重大疾病时处理、联系治疗单位急救助单位的工作。
\item 财务委员(2  个):负责处理团队的花费问题,由会计和出纳共同组成,需要做到公开透明,能省则省,一丝不苟,财务花销需要与队长商量好。
\item 装备委员(2 个):负责团队的装备统计及保存,保证装备的完整, 需要做到细心且有耐心。
\item 宣传委员(2 人):以相机或 DV 的形式纪录团队的行程、拍摄鸟 类(照片与录像),需要有敏锐的观察能力和为团队服务的精神。负责每天 QQ、微博、微信的推送,需要撰写文案。
\item 绿天使(1 个):负责安排活动过程中产生的垃圾,维持垃圾不乱丢的良好情况。
\end{enumerate}
\section*{带队队长分工与联系方式}
\section*{目的地鸟种记录}
table.....
\section*{兰州大学榆中校区鸟种记录}
table.....
\section*{兰大绿队观鸟营简介}
兰州大学绿队观鸟营是绿队暑期项目的组成部分之一,自从2008年开始,绿队就开始开展观鸟营项目,迄今为止已经成功举办了 13 届观鸟营,分别是:兰州大学绿队第一届观鸟营——关注母亲河,兰州湿地鸟类调查保护;兰州大学绿队第二届观鸟营——天水国家级保护鸟类调查保护项目;兰州大学绿队第三届观鸟营——走近白水江,关注鸟类保护。兰州大学绿队第四届观鸟营——齐聚莲花山麓,共享鸟语花香;兰州大学绿队第五届观鸟营——吐露真情,沟通自然(连城吐鲁沟鸟类调查及当地自然环境状况调研);兰州大学绿队第六届观鸟营——关注鸟类,参与保护,体验自然(定西宕昌鸟类调查保护项目);兰州大学绿队第七届观鸟营——关爱鸟类,保护自然(尕海-则岔国家级自然保护区鸟类生存境况及鸟种调查);兰州大学第八届观鸟营——关爱鸟类,共享自然(陇南宕昌官鹅沟鸟类调查项目)。兰州大学第九届观鸟营——保护鸟类,共享自然(甘肃张掖国家湿地公园——祁连山马蹄自然保护区鸟类资源和栖息地状况调查);兰州大学第十届观鸟营——甘肃省甘南藏族自治州碌曲县尕海则岔国家级自然保护区鸟种调查;兰州大学第十一届观鸟营——张掖祁连山林场马蹄寺自然保护区鸟类资源和风电站对候鸟迁徙影响的调查;兰州大学绿队第十二届观鸟营——尕海优势旅游资源及发展前景分析。兰州大学绿队第十三届观鸟营——野生动物资源调查与保护。\\
\par 兰州大学绿队观鸟营在几年的发展中不断成长,项目的范围不仅仅局限于绿队内部,而且已经走出去,跟众多兄弟社团交流合作,每年都有许多其他高校社团的成员来我校交流,参加兰大绿队的观鸟营。为了互相学习,我们也会派送绿队的成员去其他学校的观鸟营。通过这样的交流合作,全国各地的观鸟营相互取长补短,获得更多学习的机会,观鸟营也会越办越好。我们相信我们的步伐不会停止,我们的热情不会消减,只要坚持不放弃,我们观鸟、爱鸟、护鸟的团队将不断壮大。
\section*{兰大绿队介绍}
兰州大学绿队(Greenteamof Lanzhou University),又称兰州大学学生环境保护协会。成立于 1999 年 4 月 7 日,是甘肃省高校中成立较早的以加强环境保护意识宣传为中心的学生社团,如今,绿队已发展成为兰州大学规模和影响力最大的社团之一。绿队以“倡导绿色生活,维护绿色家园”为宗旨,以“人与自然和谐共处”为理念,以“从我做起,从现在做起,从小事做起”为口号,立足西北,面向全国。\\
\par 在校园里,绿队有着广泛的师生基础,一方面,绿队与学校团委保持着紧密的联系,同时积极邀请各学院老师向成员传授知识,另一方面,每年约有六百余名同学报名加入绿队,在社团中了解环保;在校外,绿队同政府组织、民间环保组织及媒体广泛接触,并和全国众多高校的环保社团建立了良好的合作关系。多年来,绿队的默默努力得到了多方面的肯定,也获得了许多荣誉。其中包括“清华大学生物多样性保护培训金奖”、“全国百佳优秀学生社团”、“中国十佳环保社团”、“根与芽年终成就奖”、“根与芽项目成就奖”、“甘肃省挑战杯一等奖”、“甘肃省优秀团队”、“全国保护母亲河一等奖”以及自成立以来连续十九年的“兰州大学优秀学生社团”等奖项。绿队的暑期社会实践活动团队屡次获“兰州大学暑期社会实践优秀团队”称号,暑期社会实践的论文成果也屡次获得“兰州大学暑期社会实践优秀论文”奖。另外,“学长的火炬——绿色爱心传递大行动”曾获全国最佳执行高校第三名,“普氏原羚保护项目(已结项)”曾获得第三届SEE•TNC生态奖(2009)环保项目奖三等奖,成为该奖项成立以来唯一一个入围并获奖的学生环保项目,2012—2013 年绿队成为绿色中国年度人物评选 24 位候选人之一。\\
\par 绿队的管理机构包括理事会、执委会、各活动部门、职能部门以及暑期项目集训队。\\
\par 多年来绿队积极申请校外基金,已先后获得过GreenSOS、GGF、清华大学BPCP生物多样性保护及阿拉善、华桥、伊山伊水等基金会的资金和其他方面的赞助。\\
\par 绿队有着丰富多彩的活动。其中校内活动有“地球一小时”、“绿色银行”、“传单回收”、“图书漂流”、“学长的火炬”、“爱鸟周——爱鸟护鸟宣传活动”、“环保井盖画”、“垃圾捡剪减”等;除此之外,绿队活动部门还有常规活动,包括环教、野营、观鸟、回收等;同时,绿队也有七大暑期项目,包括兰州大学绿色营、兰州大学观鸟营、兰州大学徒步营、西部乡村行(尖山项目转型)、绿色医者行、绿希行者公益教育项目以及前往全国各地高校社团交流的对外交流营。在网络影响力方面我们也在继续努力,绿队通过网站、微博、微信、论坛等自媒体方式宣传环保、践行环保,同时也积极与兰州各大报刊、电台进行互动交流,向更多人传播环保理念,让绿色传播得更远更广。
\end{document}
